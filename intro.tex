\chapter{Introduction}
\label{chap:intro}

Database refers to a collection of information that exists overs a long period. A Database Management System (DBMS) is a tool for creating and managing large amount of data efficiently. Early database management systems evolved from file systems. These database systems used tree-based and the graph-based models for describing the structure of the information in a database.
Database systems changed significantly following a famous paper written by Ted Codd in 1970~\cite{codd1970relational}. Codd proposed that database systems should present the user with a view of data organized as tables called relations. This paper was foundation for many relational databases like MySQL, PostgreSQL which are very popular even today.
An essential feature of relational database systems is that they provide atomicity, consistency, isolation and durability guarantees (also known as ACID properties).

\section{Organization of the thesis}
