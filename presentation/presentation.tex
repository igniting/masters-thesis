\documentclass[10pt]{beamer}

\usetheme{m}

\usepackage{booktabs}
\usepackage{fontawesome}
\usepackage{tikz}
\usetikzlibrary{arrows,positioning}
\usepackage{biblatex}
\addbibresource{citations.bib}
\usepackage{dirtree}
\usepackage{minted}

\title{Bloc: Binary Large Objects with Concurrency}
\subtitle{}
\date{19 June 2015}
\author{Anshu Avinash \\
  Thesis supervisor: Piyush Kurur}
\institute{Department of CSE, IIT Kanpur}

\begin{document}

\maketitle

\begin{frame}
  \frametitle{What is a blob?}

  BLOB stands for ``\textbf{B}inary \textbf{L}arge \textbf{OB}ject''.
  \begin{center}
    \begin{tikzpicture}[>=latex]
      \node (picture) {\Huge \faPicture};
      \node [below =1cm of picture] (film) {\Huge \faFilm};
      \node [below =1cm of film] (music) {\Huge \faMusic};
      \node [below right =1cm and 2cm of picture] (database) {\Huge \faDatabase};

      \path[draw,->] (picture) edge (database)
      (film) edge (database)
      (music) edge (database)
      ;
    \end{tikzpicture}
  \end{center}

\end{frame}

\begin{frame}[allowframebreaks]
  \frametitle{How to store blobs?}

  \textbf {Store the entire content of the blob in the database.}
  \begin{itemize}
    \item PostgreSQL breaks large objects into ``chunks'' and these chunks are
      stored in rows in the database.
    \item MongoDB also divides large objects using the GridFS specification and stores
      them in the ``chunks'' collection.
  \end{itemize}

  \framebreak

  \textbf {Store the blob in a file and store the file name in the database.}
  \begin{itemize}
    \item Number and size of blobs are now limited only by the file system.
    \item Requires to develop an interface to keep track of all the blobs of a database.
  \end{itemize}
\end{frame}

\begin{frame}
  \frametitle{Here comes Bloc!}

  We provide a library which will keep track of all the blobs stored in a database. \\

\end{frame}

\begin{frame}
  \frametitle{Design}
  \begin{itemize}[<+->]
    \item Our design is inspired from the maildir format~\footfullcite{bernstein1995using}.
    \item All the blobs of a database are stored inside a directory which we also call a \alert{BlobStore}.
    \item Each blob is stored as a different file inside the BlobStore.
    \item We provide an incremental interface for writing to a blob as well as reading from a blob.
  \end{itemize}
\end{frame}

\begin{frame}[fragile]
  \frametitle{Creating a new blob}
  \begin{minted}{haskell}
  openBlobStore :: FilePath -> IO BlobStore
  newBlob       :: BlobStore -> IO WriteContext
  \end{minted}
  \vspace{1cm}
  \pause
  \small
  \dirtree{%
    .1 \faFolderOpen \thinspace blobstore.
    .2 \faFolderOpen \thinspace tmp.
    .3 \faFile \thinspace 1a4c5091-1295-4c9c-b8d3-8e6123a51b41.
    .3 \faFile \thinspace 4af0ac03-0b79-43fd-a2b0-2cc7c62947bc.
  }
\end{frame}

\begin{frame}[fragile]
  \frametitle{Adding contents to a blob}
  \begin{minted}{haskell}
  writePartial :: WriteContext
               -> Blob
               -> IO WriteContext
  endWrite     :: WriteContext -> IO BlobId
  \end{minted}
  \vspace{1cm}
  \pause
  \small
  \dirtree{%
    .1 \faFolderOpen \thinspace blobstore.
    .2 \faFolderOpen \thinspace tmp.
    .3 \faFile \thinspace 1a4c5091-1295-4c9c-b8d3-8e6123a51b41.
    .2 \faFolderOpen \thinspace curr.
    .3 \faFileAlt \thinspace sha512-861844d6704e8573fec34d967e20b...
  }
\end{frame}

\begin{frame}[fragile]
  \frametitle{Reading from a blob}
  \begin{minted}{haskell}
  startRead   :: BlobId -> IO ReadContext
  readPartial :: ReadContext -> Int -> IO Blob
  skipBytes   :: ReadContext -> Integer -> IO ()
  endRead     :: ReadContext -> IO ()
  \end{minted}
\end{frame}

\begin{frame}
  \frametitle{Deleting blobs}
  \begin{itemize}[<+->]
    \item A blob can be shared by multiple ``records'' of the database.
    \item We provide an interface for garbage collection of deleted blobs.
  \end{itemize}
\end{frame}

\begin{frame}[fragile]
  \frametitle{Starting a GC}
  \begin{minted}{haskell}
  startGC :: BlobStore -> IO ()
  \end{minted}
  \pause
  \begin{columns}[T]
    \begin{column}{.48\textwidth}
      Before startGC
      \tiny
      \dirtree{%
        .1 \faFolderOpen \thinspace blobstore.
        .2 \faFolderOpen \thinspace tmp.
        .3 \faFile \thinspace 1a4c5091-1295-4c9c-b8d3-8e6123a51b41.
        .2 \faFolderOpen \thinspace curr.
        .3 \faFileAlt \thinspace sha512-861844d6704e8573fec34d967e20b...
        .3 \faFileAlt \thinspace sha512-ee26b0dd4af7e749aa1a8ee3c10ae...
      }
    \end{column}
    \pause
    \begin{column}{.48\textwidth}
      After startGC
      \tiny
      \dirtree{%
        .1 \faFolderOpen \thinspace blobstore.
        .2 \faFolderOpen \thinspace tmp.
        .2 \faFolderOpen \thinspace curr.
        .3 \faFileAlt \thinspace sha512-39ca7ce9ecc69f696bf7d20bb23dd...
        .2 \faFolderOpen \thinspace gc.
        .3 \faFileAlt \thinspace sha512-861844d6704e8573fec34d967e20b...
        .3 \faFileAlt \thinspace sha512-ee26b0dd4af7e749aa1a8ee3c10ae...
      }
    \end{column}
  \end{columns}
\end{frame}

\begin{frame}[fragile]
  \frametitle{Marking a blob as accessible}
  \begin{minted}{haskell}
  markAsAccessible :: BlobId -> IO ()
  \end{minted}
  \pause
  \begin{columns}[T]
    \begin{column}{.48\textwidth}
      Before
      \tiny
      \dirtree{%
        .1 \faFolderOpen \thinspace blobstore.
        .2 \faFolderOpen \thinspace tmp.
        .2 \faFolderOpen \thinspace curr.
        .3 \faFileAlt \thinspace sha512-39ca7ce9ecc69f696bf7d20bb23dd...
        .2 \faFolderOpen \thinspace gc.
        .3 \alert{\faFileAlt \thinspace sha512-861844d6704e8573fec34d967e20b..}.
        .3 \faFileAlt \thinspace sha512-ee26b0dd4af7e749aa1a8ee3c10ae...
      }
    \end{column}
    \pause
    \begin{column}{.48\textwidth}
      After
      \tiny
      \dirtree{%
        .1 \faFolderOpen \thinspace blobstore.
        .2 \faFolderOpen \thinspace tmp.
        .2 \faFolderOpen \thinspace curr.
        .3 \faFileAlt \thinspace sha512-39ca7ce9ecc69f696bf7d20bb23dd...
        .3 \alert{\faFileAlt \thinspace sha512-861844d6704e8573fec34d967e20b..}.
        .2 \faFolderOpen \thinspace gc.
        .3 \faFileAlt \thinspace sha512-ee26b0dd4af7e749aa1a8ee3c10ae...
      }
    \end{column}
  \end{columns}
\end{frame}
\begin{frame}[fragile]
  \frametitle{Ending a GC}
  \begin{minted}{haskell}
  endGC :: BlobStore -> IO ()
  \end{minted}
  \pause
  \begin{columns}[T]
    \begin{column}{.48\textwidth}
      Before
      \tiny
      \dirtree{%
        .1 \faFolderOpen \thinspace blobstore.
        .2 \faFolderOpen \thinspace tmp.
        .2 \faFolderOpen \thinspace curr.
        .3 \faFileAlt \thinspace sha512-39ca7ce9ecc69f696bf7d20bb23dd...
        .3 \faFileAlt \thinspace sha512-861844d6704e8573fec34d967e20b...
        .2 \faFolderOpen \thinspace gc.
        .3 \alert{\faFileAlt \thinspace sha512-ee26b0dd4af7e749aa1a8ee3c10ae..}.
      }
    \end{column}
    \pause
    \begin{column}{.48\textwidth}
      After
      \tiny
      \dirtree{%
        .1 \faFolderOpen \thinspace blobstore.
        .2 \faFolderOpen \thinspace tmp.
        .2 \faFolderOpen \thinspace curr.
        .3 \faFileAlt \thinspace sha512-39ca7ce9ecc69f696bf7d20bb23dd...
        .3 \faFileAlt \thinspace sha512-861844d6704e8573fec34d967e20b...
      }
    \end{column}
  \end{columns}
\end{frame}
\begin{frame}
  \frametitle{How does this design achieve concurrency?}
  \pause
  Using atomic file operations.
  \pause
  \begin{itemize}[<+->]
    \item rename
    \item mkdir
  \end{itemize}
  \vspace{1cm}
  \onslide<5>{Let us revisit our design.}
\end{frame}
\begin{frame}[fragile]
  \frametitle{Creating a new blob (Part 2)}
  \begin{minted}{haskell}
  openBlobStore :: FilePath -> IO BlobStore
  newBlob       :: BlobStore -> IO WriteContext
  \end{minted}
  \vspace{1cm}
  \small
  \dirtree{%
    .1 \faFolderOpen \thinspace blobstore.
    .2 \faFolderOpen \thinspace tmp.
    .3 \faFile \thinspace 1a4c5091-1295-4c9c-b8d3-8e6123a51b41.
    .3 \faFile \thinspace 4af0ac03-0b79-43fd-a2b0-2cc7c62947bc.
  }
\end{frame}

\begin{frame}[fragile]
  \frametitle{Adding contents to a blob (Part 2)}
  \begin{minted}{haskell}
  writePartial :: WriteContext
               -> Blob
               -> IO WriteContext
  endWrite     :: WriteContext -> IO BlobId
  \end{minted}
  \vspace{1cm}
  \small
  \dirtree{%
    .1 \faFolderOpen \thinspace blobstore.
    .2 \faFolderOpen \thinspace tmp.
    .3 \faFile \thinspace 1a4c5091-1295-4c9c-b8d3-8e6123a51b41.
    .2 \faFolderOpen \thinspace curr.
    .3 \faFileAlt \thinspace sha512-861844d6704e8573fec34d967e20b...
  }
\end{frame}

\begin{frame}[fragile]
  \frametitle{Reading from a blob (Part 2)}
  \begin{minted}{haskell}
  startRead   :: BlobId -> IO ReadContext
  readPartial :: ReadContext -> Int -> IO Blob
  skipBytes   :: ReadContext -> Integer -> IO ()
  endRead     :: ReadContext -> IO ()
  \end{minted}
\end{frame}

\begin{frame}[fragile]
  \frametitle{GC (Part 2)}
  \begin{minted}{haskell}
  startGC          :: BlobStore -> IO ()
  markAsAccessible :: BlobId -> IO ()
  endGC            :: BlobStore -> IO ()
  \end{minted}
  \vspace{0.5cm}
  \tiny
  \dirtree{%
    .1 \faFolderOpen \thinspace blobstore.
    .2 \faFolderOpen \thinspace tmp.
    .2 \faFolderOpen \thinspace curr.
    .3 \faFileAlt \thinspace sha512-39ca7ce9ecc69f696bf7d20bb23dd...
    .2 \faFolderOpen \thinspace gc.
    .3 \faFileAlt \thinspace sha512-861844d6704e8573fec34d967e20b...
    .3 \faFileAlt \thinspace sha512-ee26b0dd4af7e749aa1a8ee3c10ae...
  }
\end{frame}

\begin{frame}[fragile]
  \frametitle{Some helper functions}
  \begin{minted}{haskell}
  createBlob :: BlobStore -> Blob -> IO BlobId
  \end{minted}
  \pause
  \begin{minted}{haskell}
  readBlob :: BlobId -> IO Blob
  \end{minted}
  \pause
  \begin{minted}{haskell}
  markAccessibleBlobs :: BlobStore -> [BlobId] -> IO ()
  \end{minted}
  \pause
  \begin{minted}{haskell}
  recoverFromGC :: FilePath -> IO ()
  \end{minted}
\end{frame}

\begin{frame}
  \frametitle{Providing durability}
  \begin{itemize}[<+->]
    \item fsync a file
    \item fsync a directory
  \end{itemize}
\end{frame}

\begin{frame}
  \frametitle{Future Work}
  \begin{itemize}[<+->]
    \item Proving the correctness of our design
    \item Writing a key-value store
    \item Supporting bloc on non-POSIX systems
  \end{itemize}
\end{frame}

\end{document}
