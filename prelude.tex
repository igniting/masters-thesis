% prelude.tex
%   - titlepage
%   - dedication (optional)
%   - approval sheet
%   - course certificate
%   - table of contents, list of tables and list of figures
%   - nomenclature
%   - abstract
%============================================================================


\clearpage\pagenumbering{roman}  % This makes the page numbers Roman (i, ii, etc)



% TITLE PAGE
%   - define \title{} \author{} \date{}
\title{Bloc: Library for handling large binary objects in Haskell}
\author{Anshu Avinash}
\date{June, 2015}

%  - Roll number, required for title page, approval sheet, and
%    certificate of course work
\rollnum{10327122}

%   - The default degree is ``Doctor of Philosophy''
%     (unless the document style msthesis is specified
%      and then the default degree is ``Master of Science'')
%     Degree can be changed using the command \iitbdegree{}
\iitbdegree{Master of Technology}

%   - The default report type is preliminary report.
%      * for a PhD thesis, specify \thesis
\thesis
%      * for a M.Tech./M.Phil./M.Des./M.S. dissertation, specify \dissertation
%\dissertation
%      * for a DIIT/B.Tech./M.Sc.project report, specify \project
%\project
%      * for any other type, use  \reporttype{}
%\reporttype{ReportType}

%   - The default department is ``Unknown Department''
%     The department can be changed using the command \department{}
\department{Computer Science \& Engineering}

%    - Set the guide's name
\setguide{Piyush Kurur}
\setguidedept{Department of Computer Science \& Engineering}

%   - once the above are defined, use \maketitle to generate the titlepage
\maketitle

%--------------------------------------------------------------------%
% CERTIFICATE
%     The first page after the title page.
\makecertificate

%--------------------------------------------------------------------%
% COPYRIGHT PAGE
%   - To include a copyright page use \copyrightpage
% \copyrightpage

%--------------------------------------------------------------------%
% ABSTRACT
\begin{abstract}
  In this thesis, we describe a library for handling large binary objects (blob) written in Haskell - a purely-functional programming language. We use the idea of storing each blob as a separate file. We also try to make all the operations on a blob to be safe under concurrent access without using any locks. We leverage many features offered by Haskell like modularity and strong type system.

\end{abstract}

%--------------------------------------------------------------------%
% DEDICATION
%   Dedications, if any.
\begin{dedication}
  Dedicated to the Haskell community
\end{dedication}

% Acknowledgements
\begin{acknowledgments}
  This thesis would have been impossible without the guidance from my thesis advisor, Piyush Kurur.
  He came up with the topic of this thesis, taking into consideration my interest in Functional programming and
  systems. He was always accessible for discussions, sometimes not necessarily related to the thesis.
  I would like to express my sincere gratitude towards him.

  I would also like to thank the Haskell community (\#haskell irc channel, /r/haskell on reddit) for making
  the journey of understanding Haskell pleasant.

  I am thankful to the Department of Computer Science and Engineering, IIT Kanpur, for
  providing the necessary infrastructure for my research work.

  My family and friends always supported me throughout my thesis work - special thanks to them.
\end{acknowledgments}

%--------------------------------------------------------------------%
% CONTENTS, TABLES, FIGURES
\tableofcontents
\listoftables

\cleardoublepage
\listoffigures

\listof{program}{List of Programs}
\addcontentsline{toc}{chapter}{List of Programs}

\cleardoublepage\pagenumbering{arabic} % Make the page numbers Arabic (1, 2, etc)
