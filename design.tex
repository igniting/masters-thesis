\chapter{Design and Implementation}
\label{chap:design}

Our design is inspired from the maildir format~\cite{bernstein1995using} which stores each message in a separate file with a unique name. The mail user agent (MUA) does not have to worry about partially delivered mail: each message is safely written to disk in the \textit{tmp} subdirectory before it is moved to \textit{new}. When a mail user agent process finds message in the \textit{new} subdirectory, it moves them to \textit{curr}.

Similar to maildir, we store all blobs in separate files. All the blobs of an application are stored under a single directory which is called a ``BlobStore''.
Our library provides three operations on blobs - incremental writes, incremental reads and garbage collection (GC). We need an incremental interface for writing and reading as a blob might not fit entirely into the computer's main memory.
Instead of providing a method for deleting the blob directly, we provide an interface for GC of all the deleted blobs, since the same blob might be shared by multiple records of a database.

In this chapter, we discuss implementation of these three operations. We also discuss how our implementation provides a lock free concurrent access on blobs.

\section{Initializing the BlobStore}
All the new blobs of a BlobStore are created inside a subdirectory named \textit{tmp}. After doing a series of incremental writes on a newly created blob, it is moved to another subdirectoy named \textit{curr}. Hence, before starting any operations in a BlobStore, we ensure that the \textit{tmp} and \textit{curr} subdirectories have already been created.
We provide a method \texttt{openBlobStore} which takes the path of a directory which is to be used as BlobStore as argument and does the initialization for us.
BlobStore is defined as a Haskell newtype. We don't expose the constructor for BlobStore, so the only way to get a BlobStore is by using the \texttt{openBlobStore} method.

\begin{program}
  \caption{Definition of BlobStore}
  \label{prog:defblobstore}
  \inputminted{haskell}{hs/blobstore.hs}
\end{program}

The \texttt{openBlobStore} method also creates a file named ``metadata'' in the \textit{curr} directory. Currently this file just contains the version
number of our library. Creation on this file ensures that the \textit{curr} directory is never empty.

We use Haskell's \texttt{createDirectory} method to create \textit{tmp} and \textit{curr}, which internally uses \texttt{mkdir} system call.
The POSIX specification of \texttt{mkdir}~\cite{mkdirposix} says that if the directory to be created by \texttt{mkdir} already exists, ~\texttt{mkdir} should fail with EEXIST error.
This guarantee makes concurrent calls to \texttt{openBlobStore} safe. If two processes try to call \texttt{openBlobStore} at the same time - only one of them will be able to create \textit{tmp} directory and similarly only one of them will be able to create the \textit{curr} directory.
This ensures that any data present in \textit{tmp} or \textit{curr} is never lost by a call to \texttt{openBlobStore}.

The \texttt{openBlobStore} method also does the recovery from a failed garbage collection. This is explained in more detail in the section \ref{recovercrashedgc}.

\section{Creating a new blob}
We provide a method called \texttt{newBlob} for creating a new blob. It takes a BlobStore as a parameter and returns a value of type \texttt{WriteContext}.

\begin{program}
  \caption{Definition of WriteContext}
  \label{prog:defwritecontext}
  \inputminted{haskell}{hs/writecontext.hs}
\end{program}

WriteContext contains the file handle of just created blob, a value of type \texttt{TempLocation} and a hash context. The \texttt{TempLocation} type stores the base directory and the filename of the newly created blob. The hash context is used to store the SHA-512 hash of the contents that has been written to the blob during the incremental writes.

All the new blobs are created in the \textit{tmp} directory. We use Version 4 UUID~\cite{leach2005universally} to give unique names to the newly created blobs.
Creation of new blobs is concurrent in the sense that if two processes call \texttt{newBlob} on a same BlobStore, both will be able to create new blobs successfully if the \textit{tmp} directory is always present. We had ensured the presence of \textit{tmp} directory in the \texttt{openBlobStore} method. Also, \textit{tmp} directory is never deleted by any of our operations.

\section{Writing to a blob}
We only allow to add new data at the end of a given blob. The function \texttt{writePartial} takes the new data and a value of type \texttt{WriteContext} as arguments and appends the new data to the blob.

Once all the data has been written to a blob, we call the \texttt{endWrite} method. The \texttt{endWrite} method takes a value of type \texttt{WriteContext} as argument and moves the blob from \textit{tmp} directory to \textit{curr} directory. Before moving the blob, we ensure that all the data has been written to disk by calling the \texttt{fdatasync} system call.
We also call \texttt{fdatasync} on the \textit{curr} directory to ensure that even in a case of crash, the \textit{curr} directory contains the entry of the moved file.

In the \texttt{endWrite} method, we also rename the file to SHA-512 hash of its contents. Using hash of the contents as filename ensures that if multiple blobs have same contents, only one copy is stored. This also provides the benefit of verifying the file content at a later time - compute the SHA-512 hash of the contents and if does not match with the file name, data has been corrupted.
We prefix the filename with the name of the hash function i.e. ``sha512''. This is for future compatibility, in case we want to support other hash functions.

\begin{figure}[hbt]
  \caption{Directory structure of a BlobStore}
  \label{fig:blobstore-dirstructure}
  \dirtree{%
    .1 blobstore.
    .2 tmp.
    .3 f66affb7-ad10-4583-8986-c4a6892d0120.
    .2 curr.
    .3 sha512-a75ebf9a0f109288d3eae1ecbfd89....
    .3 metadata.
  }
\end{figure}

The \texttt{endWrite} method returns a value of type \texttt{BlobId} which contains the location of the blob. No more updates to the blob are possible after calling \texttt{endWrite}.

\begin{program}
  \caption{Definition of BlobId}
  \label{prog:defblobid}
  \inputminted{haskell}{hs/blobid.hs}
\end{program}

In the \texttt{endWrite} method, we move the blob to \textit{curr} directory using the \texttt{rename} system call, which is atomic~\cite{renamemanpage}. Before calling \texttt{rename}, we have to ensure that the \textit{curr} directory is present. We had created the \textit{curr} directory in the call to \texttt{openBlobStore}.
However, the \texttt{startGC} method described in \ref{startgc} renames the \textit{curr} directory to \textit{gc} directory and then creates an empty \textit{curr} directory. During this short interval of renaming \textit{curr} to \textit{gc} and creation of new \textit{curr}, all calls to \texttt{rename} in \texttt{endWrite} will fail.
We solve this problem by taking the optimistic approach - we retry calling \texttt{rename} in \texttt{endWrite}. The maximum number of such retries can be specified by the user.

\section{Reading from blob} \label{readblob}
Similar to writes, reading of a blob is also incremental. First the \texttt{startRead} method is called which takes a value of type \texttt{BlobId} as argument and returns a value of type \texttt{ReadContext}. The \texttt{ReadContext} type  contains the file handle of the blob which is opened in read mode.

\begin{program}
  \caption{Definition of ReadContext}
  \label{fig:defreadcontext}
  \inputminted{haskell}{hs/readcontext.hs}
\end{program}

The \texttt{readPartial} method takes a value of type \texttt{ReadContext} and number of bytes as input and returns those number of bytes from the blob.
While reading from a blob, you can skip ahead using the method \texttt{skipBytes}. The method \texttt{skipBytes} takes a value of type \texttt{ReadContext} and number of bytes, \textit{b} as input and skips \textit{b} bytes ahead in the blob.
The \texttt{endRead} method closes the handle which was opened by \texttt{startRead}.

Usually the blob to be read is present in the \textit{curr} directory. However, during a GC, the blob might either be present in \textit{curr} or in \textit{gc} (Refer section \ref{garbagecollection} for details on GC). The \texttt{startRead} method has to do three lookups in the worst case. It first tries to read from the \texttt{curr} directory.
At this time, if the blob is not found in \texttt{curr} directory, it might be in the \texttt{gc} directory. However the process which is running GC might move the blob back to \texttt{curr} before \texttt{startRead} reads from GC. The final lookup happens in the \texttt{curr} directory.

\begin{table}
\caption{Interface for operations on blob}
\label{tab:interface-blob}
\begin{center}
  \begin{tabularx}{0.91\textwidth}{lX}
    \hline\noalign{\smallskip}
    Method & Purpose \\
    \noalign{\smallskip}
    \hline
    \noalign{\smallskip}
    \texttt{openBlobStore} & Initializes given directory to be used as a BlobStore \\
    \texttt{newBlob} & Creates a blob in the given BlobStore\\
    \texttt{writePartial} & Takes data as input and appends it at the end of the blob given in the argument\\
    \texttt{endWrite} & Completes the write on the given blob \\
    \texttt{startRead} & Initializes the read of the given blob \\
    \texttt{readPartial} & Reads a given number of bytes from the given blob \\
    \texttt{skipBytes} & Skips ahead a given number of bytes in the given blob \\
    \texttt{endRead} & Completes the read \\
    \hline
  \end{tabularx}
\end{center}
\end{table}

\begin{figure}
  \caption{Order of operations on blob}
  \label{fig:blob-operations-order}
  \begin{tikzpicture}[>=latex]
    \tikzset{ block/.style= {draw, rectangle, align=center, minimum width=3cm, minimum height=1cm}
    },
    \node [block] (openBlobStore) {openBlobStore};
    \node [block, right =1.5cm of openBlobStore] (newBlob) {newBlob};
    \node [block, right =1.5cm of newBlob] (writePartial) {writePartial};
    \node [block, right =1.5cm of writePartial] (endWrite) {endWrite};
    \node [block, below =2.5cm of endWrite] (startRead) {startRead};
    \node [block, above left =0.5cm and 1.5cm of startRead] (readPartial) {readPartial};
    \node [block, below left =0.5cm and 1.5cm of startRead] (skipBytes) {skipBytes};
    \node [block, below left =0.5cm and 1.5cm of readPartial] (endRead) {endRead};

    \path[draw,->] (openBlobStore) edge (newBlob)
                   (newBlob) edge (writePartial)
                   (writePartial) edge [loop above] (writePartial)
                   (writePartial) edge (endWrite)
                   (endWrite) edge (startRead)
                   (startRead) edge (readPartial)
                   (startRead) edge (skipBytes)
                   (readPartial) edge [loop above] (readPartial)
                   (skipBytes) edge [bend right] (readPartial)
                   (readPartial) edge [bend right] (skipBytes)
                   (readPartial) edge (endRead)
                   ;
  \end{tikzpicture}
\end{figure}

\section{Helper methods for small blobs}
Blobs which can easily fit into the main memory, do not require the incremental write and read methods. For such blobs, we provide helper methods \texttt{createBlob} and \texttt{readBlob}.

The \texttt{createBlob} method takes a BlobStore and the blob contents as arguments and returns the BlobId of the newly created blob. It basically combines the \texttt{newBlob}, \texttt{writePartial} and \texttt{endWrite} methods.

\begin{program}
  \caption{Definition of createBlob}
  \label{fig:defcreateblob}
  \inputminted{haskell}{hs/createblob.hs}
\end{program}

The \texttt{readBlob} method takes a value of type \texttt{BlobId} as argument and returns its entire contents.

\begin{table}[hbt]
\caption{Helper methods for small blobs}
\label{tab:interface-small-blob}
\begin{center}
  \begin{tabularx}{0.91\textwidth}{lX}
    \hline\noalign{\smallskip}
    Method & Purpose \\
    \noalign{\smallskip}
    \hline
    \noalign{\smallskip}
    \texttt{createBlob} & Creates a blob with a given content in a BlobStore \\
    \texttt{readBlob} & Reads the entire contents of the given blob\\
    \hline
  \end{tabularx}
\end{center}
\end{table}


\section{Garbage Collection} \label{garbagecollection}
It is quite likely that the same blob would be shared by multiple ``values'' in the database. For a relational database these values are rows in a table, while for a document-oriented database, these values are documents.
Hence, we provide an interface for garbage collecting the deleted blobs.

\subsection{Starting the Garbage Collection} \label{startgc}
The \texttt{startGC} method takes a BlobStore as argument and starts GC for that BlobStore.
The \texttt{startGC} method does three things:
  \begin{enumerate}
      \item Renames the \textit{curr} directory to \textit{gc}. \\
        We use the \texttt{rename} system call to atomically move \textit{curr} to \textit{gc}. If a \textit{gc} directory is already present, \texttt{rename} will fail with EEXIST or ENOTEMPTY error~\cite{renamemanpage}. This ensures that we have only one GC running at any time.
      \item Creates a new \textit{curr} directory. \\
        We need to ensure that the \texttt{endWrite} method is able to move files to the \texttt{curr} directory.
      \item Copy the metadata file from \textit{gc} to \textit{curr}. \\
  \end{enumerate}

\begin{figure}[hbt]
  \caption{Directory structure of a BlobStore during GC}
  \label{fig:blobstore-dirstructure-gc}
  \dirtree{%
    .1 blobstore.
    .2 tmp.
    .3 1a4c5091-1295-4c9c-b8d3-8e6123a51b41.
    .2 gc.
    .3 sha512-11853df40f4b2b919d3815f64792e....
    .3 metadata.
    .2 curr.
    .3 sha512-9b71d224bd62f3785d96d46ad3ea3....
    .3 metadata.
  }
\end{figure}

\subsection{Marking a blob as accessible}
Once a blob is marked as not deleted using the method \texttt{markAsAccessible}, we move the blob from the \textit{gc} directory to the \textit{curr} directory.
We use the \texttt{rename} system call, which is atomic.
We also call \texttt{fdatasync} system call on the file descriptor of the \textit{curr} directory to ensure that the \textit{curr} directory contains an entry for the moved blob, even in case of a system crash.

\subsection{End Garbage collection}
This step involves removal of all the blobs which are not accessible. The \texttt{endGC} method takes a BlobStore as argument. It first deletes all the blobs in the \textit{gc} directory as well as the ``metadata'' file.

The \texttt{endGC} method then waits for a user specified interval before deleting the \textit{gc} directory. This is done to restrict the number of lookups during \texttt{startRead} to 3 (Refer section \ref{readblob}). If we remove this wait, another GC can start and move back the blob which is being looked up \texttt{startRead} back to the \texttt{gc} directory.

\subsection{Marking a list of blobs as accessible}
Calling \texttt{endGC} without marking all the accessible blobs would lead to their deletion. To prevent such accidental calls, we provide a method \texttt{markAccessibleBlobs} which takes a list of BlobId as an argument. After calling \texttt{startGC}, \texttt{markAccessibleBlobs} marks all the blobs in input list as accessible and finally calls \texttt{endGC}.
This method should be used only when the number of accessible blobs is small.

\begin{table}[hbt]
\caption{Interface for garbage collection}
\label{tab:interface-gc}
\begin{center}
  \begin{tabularx}{0.91\textwidth}{lX}
    \hline\noalign{\smallskip}
    Methods & Purpose \\
    \noalign{\smallskip}
    \hline
    \noalign{\smallskip}
    \texttt{startGC} & Starts garbage collection for the given BlobStore\\
    \texttt{markAsAccessible} & Marks the given blob as accessible\\
    \texttt{endGC} & Ends the garbage collection by removing all the unaccessible blobs\\
    \hline
  \end{tabularx}
\end{center}
\end{table}

\subsection{Recovering from a crashed GC} \label{recovercrashedgc}
A process which had started GC might crash before calling \texttt{endGC}. In this case no other process would be able to start a new GC as explained in section \ref{startgc}.
To handle this, we always call \texttt{recoverFromGC} in \texttt{openBlobStore} method. The method \texttt{recoverFromGC} moves all the blobs from the \texttt{gc} directory back to the \texttt{curr} directory and deletes the \texttt{gc} directory.
Note that \texttt{recoverFromGC} will nullify the effect of a GC which is currently running since it moves all blobs back to the \texttt{curr} directory - irrespective of whether they have been deleted or not.
