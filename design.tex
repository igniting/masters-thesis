\chapter{Design}
\label{chap:design}

Our design is inspired from the maildir format~\cite{bernstein1995using}. maildir format stores each message in a separate file with a unique name. All the partially delivered mails are stored in the \textit{tmp} subdirectory before it is moved to \textit{new}. When a mail user agent process finds message in the \textit{new} subdirectory, it moves them to \textit{cur}.

Similar to maildir, we also store all large objects in separate files. All the large objects of a database are stored under a single directory which we also call a ``BlobStore''.
The BlobStore contains three subdirectories: \textit{tmp}, \textit{curr} and \textit{old}. We will discuss purpose of these directories later in this chapter.

\section{Initializing the BlobStore}
Before starting to create blobs inside a directory, we ensure that the \textit{tmp} and \textit{curr} subdirectories have already been created. We provide a method \texttt{initBlobStore} which takes the path of a directory which is to be used as BlobStore as argument and does the initialization for us.

\section{Creating a Blob}
We provide a method called \texttt{createBlob} for creating a new blob. It takes a BlobStore as a parameter and returns a WriteContext. WriteContext contains the file handle of just created blob among other things. All the new blobs are created in the \textit{tmp} folder. We use Version 4 UUID~\cite{leach2005universally} to give unique names to the newly created blobs.

\section{Writing to a Blob}
We only allow to add new data at the end of a given blob. We provide \texttt{writePartial} method for this. \texttt{writePartial} takes a blob and a WriteContext as arguments and appends the given blob to the WriteContext's blob.
Once all the data has been written to the blob, \texttt{finalizeWrite} is called. \texttt{finalizeWrite} takes a WriteContext as argument and moves the blob from \textit{tmp} folder to \textit{curr} folder. We also rename the file to SHA-512 hash of its contents.
\texttt{finalizeWrite} returns a BlobId. This BlobId contains the location of the blob. No more updates to the blob are possible after calling \texttt{finalizeWrite}.

\section{Reading from Blob}
Reading is also sequential. First the \texttt{initRead} method is called which returns a ReadContext, similar to the WriteContext. ReadContext also contains the file handle of the blob which is opened in read mode.
\texttt{readPartial} takes a ReadContext and number of bytes as input and returns those number of bytes from the blob.

\begin{table}[hbt]
\caption{Interface for operations on blob}
\label{tab:interface-blob}
\begin{center}
  \begin{tabularx}{0.91\textwidth}{lX}
    \hline\noalign{\smallskip}
    Methods & Purpose \\
    \noalign{\smallskip}
    \hline
    \noalign{\smallskip}
    \texttt{initBlobStore} & Initializes given directory to be used as a BlobStore \\
    \texttt{createBlob} & Creates a blob in the given BlobStore\\
    \texttt{writePartial} & Takes a blob and appends it to the end of the blob given in the argument\\
    \texttt{finalizeWrite} & Takes a WriteContext as input and returns a BlobId \\
    \texttt{initRead} & Takes a BlobId as input and returns a ReadContext \\
    \texttt{readPartial} & Reads a given number of bytes from a Blob \\
    \texttt{finalizeRead} & Completes the read \\
    \hline
  \end{tabularx}
\end{center}
\end{table}

\section{Garbage Collection}
It is quite likely that the same blob would be shared by multiple ``values'' in the database. For a relational database these values are rows in a table, while for a document-oriented database, these values are documents.
Hence, we provide an interface for garbage collecting the deleted documents.

\subsection{Starting the Garbage Collection}
The \texttt{startGC} method takes a BlobStore as argument and starts garbage collection (GC) for that BlobStore.
\texttt{startGC} does two things: It first renames the \textit{curr} folder to \textit{old} and then creates an empty \textit{curr} folder.
Once a GC has started you can not start another GC on the same BlobStore until the first one finishes - doing so will throw an error. Also, note that creation of new blobs and reading the old blobs can happen concurrently with the GC.

\subsection{Marking a blob as accessible}
Once a blob is marked as not deleted using the method \texttt{markBlobAsAccessible}, we move it from the \textit{old} folder to the \textit{curr} folder. This ensures that the blob does not get deleted at the end of the GC.

\subsection{End Garbage collection}
This step involves removal of all the blobs which are not accessible. The \texttt{endGC} method takes a BlobStore as argument and delete the \textit{old} subdirectory along with its contents.

\begin{table}[hbt]
\caption{Interface for garbage collection}
\label{tab:interface-gc}
\begin{center}
  \begin{tabularx}{0.91\textwidth}{lX}
    \hline\noalign{\smallskip}
    Methods & Purpose \\
    \noalign{\smallskip}
    \hline
    \noalign{\smallskip}
    \texttt{startGC} & Starts garbage collection for the given BlobStore\\
    \texttt{markBlobAsAccessible} & Marks the given blob as accessible\\
    \texttt{endGC} & Ends the garbage collection by removing all the unaccessible blobs\\
    \hline
  \end{tabularx}
\end{center}
\end{table}
